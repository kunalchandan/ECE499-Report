\section{Phase 1 - Uniform Bonding}

\subsection{Characterization of Indium Bonding Spread}
After a repeatable electroplating process for indium was developed, the next step was to investigate how the indium would spread once die-bonded. The spreading behaviour of indium is critical because it can impact the performance of the semiconductor device. Specifically, the spreading of indium can affect the heat dissipation and the electrical contact between the die and the substrate. Understanding this behaviour is essential for optimizing the performance of the device.
This knowledge can then be used to inform the design and manufacture of semiconductor devices, improving their reliability and performance.
Diebonding is the process of placing a semiconductor chip onto a substrate or package, and eutectic bonding is a common diebonding technique. Understanding the behaviour of indium during eutectic bonding is crucial for ensuring the reliability and performance of the final device.

A flipchip diebonder is a machine that is used to bond microelectronic chips directly onto a substrate. It allows for precise alignment and bonding of the chip to the substrate, which is essential for the proper functioning of microelectronic devices. In addition to the standard bonding techniques, a flipchip diebonder with eutectic bonding capabilities enables the bonding of two chips by bonding with precise control of heat and pressure, which can greatly enhance the reliability and performance of the microelectronic device. Eutectic bonding is a specialized technique that involves heating the two materials to their eutectic point, at which they melt and fuse together, creating a strong and reliable bond. This makes the flipchip diebonder with eutectic bonding capability a valuable tool in the field of microelectronics for high-performance and reliable bonding of chips to substrates.


The diebonder being used is the `TRESKY-diebond' tool in the QNC packaging lab it is the Tresky T-3000-FC3 model.
The tool is capable of providing a bonding force of up to $490 \unit{\newton}$ ($50 \unit{\kg}$ mass) may be applied at temperatures up to $400\unit{\degreeCelsius}$. The datasheet suggests that it offers placement accuracy of $10\um$ \cite{diebonderDatasheet}.



Insert calculation of bonding volume
Diebonding process fits here as well

\subsection{Thermal Simulations}

Insert figures about the thermal simulations and refer to the code that is in the appendix

Discuss the lack of wetting to the LEDs. insert the sem images and the EDX images

\subsection{Conformation to Theory}
Conformed to theory when the bonding would spread according to the calculated volume

\subsection{Development of a Repeatable Process}
% Adding anchors