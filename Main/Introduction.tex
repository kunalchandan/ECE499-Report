\section{Introduction}

LEDs are exceptionally efficient when compared to legacy lighting technologies like arc, incandescent, fluorescent lighting, and others. The advantages inherent to the technology have allowed LEDs to enter a variety of light applications like automotive, general lighting, and display backlighting and many other use-cases. \cite{uLED_review}
Conventional inorganic LEDs have MESA dimensions generally greater than $(300\times 300) \unit{\micro\meter\squared}$, however \uled s target an area below $(100\times 100) \unit{\micro\meter\squared}$ to $1 \unit{\micro\meter\squared}$ \cite{parbrook2021micro}

One of the earliest claims to the discoveries of the LED was by Oleg Losev in the 1920s, the work was generally ignored and the conjectured theory for the operation was incorrect, the subsequent research also focused on SiC and \rom{2}-\rom{4} semiconductors. This era of research was generally impractical and did not produce sufficient light. However, with the arrival of \rom{3}-\rom{5} semiconductors like GaAs, GaSb, InP, and SiGe there was significant progress in luminosity although sadly this was all within the infrared spectrum. 
Visible LEDs would emerge as research in the area quickened, the technology was based on GaAsP epitaxy over GaAs substrates. This would bring forth the advent of commercializable LEDs that would now be seen everywhere. Eventually efficiency and luminosity would surpass that of traditional lighting solutions like filament based tungsten and bring us to where we are today with LEDs being used in nearly all lighting and display applications.




\subsection{Motivation for \uled s as a Technology}
LCDs and OLEDs currently dominate the display market, each technology comes with its trade-offs and current innovations with quantum dots and fully addressable back-lit mini-LED panels aim to address some of the shortcomings concerning contrast, colour accuracy, power consumption, brightness, lifetime, and response times.

\uled s aim to address many of these concerns as well, by offering a number of advantages over traditional LCD and OLED displays. The biggest advantage \uled s would provide over LCDs is the power effeciency where LCDs suffer from high power loss due to the multiple diffuser and filtering layers required to compose a screen (CITE AND REWORD). A commonly cited number is that LCDs loose nearly 70-90\% of the flux introduced by the backlight to the various polarizing layers that comprise the display. In contrast as \uled  displays would be entirely emissive, none of the light would be lost to the conventional filtering layers. (IS THIS EVEN TRUE, IS LIGHT LOST TO PHOSPHOR LAYERS?)


\subsection{Hurdles in \uled Technology}

IDENTIFY COMMON HURDLES IN MICRO LED TECHNOLOGY

SHOW WHICH OF THESE PROBLEMS WE ARE AIMING TO FIX OR SOLVE FOR


\subsection{Motivation for Indium as a Diebonding Material}

EASE OF INDIUM AS AN ELECTROPLATING MATERIAL

LOW MELTING POINT

SOFTNESS OF MATERIAL/DUCTILITY

LOW LIKELYHOOD OF SURFACE OXIDES

EASE OF WETTING TO SURFACE
