
%% Custom Commands
\definecolor{purple1}{RGB}{208, 180, 239}
\definecolor{purple2}{RGB}{190, 051, 218}
\definecolor{purple3}{RGB}{129, 000, 180}
\definecolor{purple4}{RGB}{087, 005, 139}

% Roman numerals \rom{2} = II
\makeatletter
\newcommand*{\rom}[1]{\expandafter\@slowromancap\romannumeral #1@}
\makeatother

% microLED
\newcommand{\uled}{$\mu$LED}
\newcommand{\uleds}{$\mu$LEDs}

% Units
\DeclareSIUnit\angstrom{\text {Å}}
\newcommand{\umm}   {\unit[per-mode=fraction]{\mu\meter\squared}}
\newcommand{\um}    {\unit[per-mode=fraction]{\micro\meter}}
\newcommand{\nm}    {\unit[per-mode=fraction]{\nano\meter}}
\newcommand{\mm}    {\unit[per-mode=fraction]{\milli\meter}}
\newcommand{\mA}    {\unit[per-mode=fraction]{\milli\ampere}}
\newcommand{\mV}    {\unit[per-mode=fraction]{\milli\volt}}
\newcommand{\V}     {\unit[per-mode=fraction]{\volt}}
\newcommand{\mpa}   {\unit[per-mode=fraction]{\mega\pascal}}
\newcommand{\sqmm}  {\unit[per-mode=fraction]{\square\milli\meter}}
\newcommand{\junits}{\unit[per-mode=fraction]{\milli\ampere\per\square\milli\meter}}
\newcommand{\jufeet}{\unit[per-mode=fraction]{\ampere\per ft^2}}
\newcommand{\mmph}  {\unit[per-mode=fraction]{\mm\per\hour}}
\newcommand{\umpm}  {\unit[per-mode=fraction]{\um\per\minute}}
\newcommand{\dC}    {\unit[per-mode=fraction]{\degreeCelsius}}




\makeatletter
%% Minipage
% #1 Name
% #2 Indent
% #3 Beforeskip
% #4 Afterskip
% #5 Style
% #6 Heading
\newcommand*{\@mpstartsection}[6]{%
  \if@noskipsec \leavevmode \fi
  \par
  \@tempskipa #3\relax
  \@afterindenttrue
  \ifdim \@tempskipa <\z@
    \@tempskipa -\@tempskipa \@afterindentfalse
  \fi
  \if@nobreak
    \everypar{}%
  \else
    \addvspace\@tempskipa
  \fi
  \refstepcounter{cont@#1}% this counter is never reset, used for hyperref
  \stepcounter{#1}% counter used for display purposes
  \protected@edef\@currentlabel{% define how refs look
    \csname p@#1\endcsname \csname the#1\endcsname
  }%
  \protected@edef\@svsec{\@seccntformat{#1}\relax}%
  \begingroup
    #5{\@hangfrom{\hskip #2\relax\@svsec}#6\@@par}%
  \endgroup
  \par
  \vskip #4\relax
  % \@afterheading deals with penalities related to page breaking that aren't
  % needed inside a minipage, but it also honors the \if@afterindent switch,
  % therefore leave it here.
  \@afterheading
  \ignorespaces
}

\newcounter{mpsection}
\newcounter{mpsubsection}[mpsection]
\newcounter{cont@mpsection}
\newcounter{cont@mpsubsection}
\renewcommand\thempsection{\@arabic\c@mpsection}
\renewcommand\thempsubsection{\thempsection.\@arabic\c@mpsubsection}
\AtBeginEnvironment{minipage}{%
  \setcounter{mpsection}{0}%
}
\newcommand\mpsection{\@mpstartsection{mpsection}{\z@}{-3.5ex \@plus -1ex \@minus -.2ex}{2.3ex \@plus.2ex}{\normalfont\normalsize\bfseries}}
\newcommand\mpsubsection{\@mpstartsection{mpsubsection}{\z@}{-3.25ex \@plus -1ex \@minus -.2ex}{1.5ex \@plus .2ex}{\normalfont\normalsize\itshape}}


\makeatletter
\def\fillRGB#1#2#3{\pgfsys@color@rgb@fill{#1}{#2}{#3}}
\makeatother
\definecolor{purple4}{RGB}{087, 005, 139}
%======================================
% Color set related!
\definecolor{high}{HTML}{57058B}  % the color for the highest number in your data set
\definecolor{low}{HTML}{FFFFFF}  % the color for the lowest number in your data set
\newcommand*{\opacity}{70}% here you can change the opacity of the background color!
%======================================
% Data set related!
\newcommand*{\minval}{2.61}% define the minimum value on your data set
\newcommand*{\maxval}{18.1}% define the maximum value in your data set!

%======================================
% gradient function!
\newcommand{\gradient}[1]{
    % The values are calculated linearly between \minval and \maxval
    \ifdimcomp{#1pt}{>}{\maxval pt}{#1}{
        \ifdimcomp{#1pt}{<}{\minval pt}{#1}{
            \pgfmathparse{int(round(100*(#1/(\maxval-\minval))-(\minval*(100/(\maxval-\minval)))))}
            \xdef\tempa{\pgfmathresult}
            \cellcolor{high!\tempa!low!\opacity} #1
    }}
}
\newcommand{\grd}[1]{\gradient{#1} }
%======================================
% gradient function single cell!
\newcommand{\gradientcell}[6]{
    % The values are calculated linearly between \minval and \maxval
    \ifdimcomp{#1pt}{>}{#3 pt}{#1}{
        \ifdimcomp{#1pt}{<}{#2 pt}{#1}{
            \pgfmathparse{int(round(100*(#1/(#3-#2))-(\minval*(100/(#3-#2)))))}
            \xdef\tempa{\pgfmathresult}
            \cellcolor{#5!\tempa!#4!#6} #1
    }}
}

\makeatletter
\renewcommand\chapter{\if@openright\cleardoublepage\else\clearpage\fi
                    \thispagestyle{fancy}% original style: plain
                    \global\@topnum\z@
                    \@afterindentfalse
                    \secdef\@chapter\@schapter}
\makeatother