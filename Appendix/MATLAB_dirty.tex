\definecolor{dkgreen}{rgb}{0,0.6,0}
\definecolor{gray}{rgb}{0.5,0.5,0.5}
\definecolor{mauve}{rgb}{0.58,0,0.82}

\lstset{frame=tb,
  language=MATLAB,
  aboveskip=3mm,
  belowskip=3mm,
  showstringspaces=false,
  columns=flexible,
  basicstyle={\small\ttfamily},
  numbers=none,
  numberstyle=\tiny\color{gray},
  keywordstyle=\color{blue},
  commentstyle=\color{dkgreen},
  stringstyle=\color{mauve},
  breaklines=true,
  breakatwhitespace=true,
  tabsize=3
}
\begin{lstlisting}
%% Define geometry constants
at = 1.1; % aligator clip thickness mm
al = 5; % Length of the aligator clip that sticks out mm
als3 = al*sqrt(3); % assuming aligator clip is at 30deg mm
st = 0.5; % Sample thickness mm
cld = 2; % How much the aligator clip is attached to the sample mm
s_len = 10; % Length of sample mm

scaling = 1/1000;

%% Define geometries
aligator_back_x = scaling*[0 0 0 at at at]; % mm
aligator_back_y = scaling*[0 (als3) (als3+cld) (als3+cld) als3 0]; % mm

aligator_front_x = scaling*[(at+st+al) (at+st+al+at) (at+st+at) (at+st+at) (at+st) (at+st)]; % mm
aligator_front_y = scaling*[0 0 als3 (als3+cld) (als3+cld) als3]; % mm

sample_x = scaling*[at (at+st) (at+st) at]; % mm
sample_y = scaling*[als3 als3 (als3+s_len) (als3+s_len)]; % mm

in_electrode_x = scaling*[(at+50) (at+52) (at+52) (at+50)];
in_electrode_y = scaling*[-5 -5 (50-5) (50-5)];
%% Draw in PDE solver
pdepoly(aligator_back_x, aligator_back_y, 'Aligator_Back');
pdepoly(aligator_front_x, aligator_front_y, 'Aligator_front');
pdepoly(sample_x, sample_y, 'Sample');
pdepoly(in_electrode_x, in_electrode_y, 'Electrode');
pderect([-0.06 0.08 -0.04 0.08], "Conductive_Liquid");

%% Generate PDE Plot
% Export the geometry description matrix, set formula, and name-space matrix into the MATLAB workspace by selecting
%%%%%%%% Draw > Export Geometry Description, Set Formula, Labels.
% This data lets you reconstruct the geometry in the workspace.

g = decsg(gd,sf,ns);

%% Create E Model
emagmodel = createpde("electromagnetic", "conduction");

geometryFromEdges(emagmodel, g)
%% Rest

pdegplot(emagmodel,"EdgeLabels","on", "FaceLabels", "on")

emagmodel.VacuumPermittivity = 8.8541878128E-12;
electromagneticProperties(emagmodel, ...
    "RelativePermittivity",1.00059, ...
    "Conductivity", 1);
%%
electromagneticBC(emagmodel,"Voltage",1.1,"Edge",[6 7 19 20]);
electromagneticBC(emagmodel,"Voltage",0.5,"Edge",[8 9 23 24]);
electromagneticBC(emagmodel,"Voltage",0,"Edge",[1 2 3 4 5 10 11 12 13 14 15 16 17 18 21 22]);


generateMesh(emagmodel);

R = solve(emagmodel);

pdeplot(emagmodel,"XYData",R.ElectricPotential, ...
    "Contour","on", ...
    "FlowData",[R.CurrentDensity.Jx,R.CurrentDensity.Jy])

ali_back_poly = polyshape(aligator_back_x, aligator_back_y);
ali_front_poly = polyshape(aligator_front_x, aligator_front_y);
sample_poly = polyshape(sample_x, sample_y);
in_poly = polyshape(in_electrode_x, in_electrode_y);

hold on;
plot(ali_back_poly);
plot(ali_front_poly);
plot(sample_poly);
plot(in_poly);
axis equal
hold off;
\end{lstlisting}
