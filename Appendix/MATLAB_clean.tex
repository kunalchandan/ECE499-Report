\definecolor{dkgreen}{rgb}{0,0.6,0}
\definecolor{gray}{rgb}{0.5,0.5,0.5}
\definecolor{mauve}{rgb}{0.58,0,0.82}

\lstset{frame=tb,
  language=MATLAB,
  aboveskip=3mm,
  belowskip=3mm,
  showstringspaces=false,
  columns=flexible,
  basicstyle={\small\ttfamily},
  numbers=none,
  numberstyle=\tiny\color{gray},
  keywordstyle=\color{blue},
  commentstyle=\color{dkgreen},
  stringstyle=\color{mauve},
  breaklines=true,
  breakatwhitespace=true,
  tabsize=3
}
\begin{lstlisting}
  %% Define geometry constants
  at = 1.1; % aligator clip thickness mm
  al = 5; % Length of the aligator clip that sticks out mm
  als3 = al*sqrt(3); % assuming aligator clip is at 30deg mm
  st = 0.5; % Sample thickness mm
  cld = 2; % How much the aligator clip is attached to the sample mm
  s_len = 10; % Length of sample mm

  scaling = 1/1000;

  %% Define geometries
  sample_x = scaling*[at (at+st) (at+st) at]; % mm
  sample_y = scaling*[als3 als3 (als3+s_len) (als3+s_len)]; % mm

  in_electrode_x = scaling*[(at+50) (at+52) (at+52) (at+50)];
  in_electrode_y = scaling*[-5 -5 (50-5) (50-5)];

  pdepoly(sample_x, sample_y, 'Sample');
  pdepoly(in_electrode_x, in_electrode_y, 'Electrode');
  pderect([-0.06 0.08 -0.04 0.08], "Conductive_Liquid");

  %% Generate PDE Plot
  % Export the geometry description matrix, set formula, and name-space matrix into the MATLAB workspace by selecting
  %%%%%%%% Draw > Export Geometry Description, Set Formula, Labels.
  % This data lets you reconstruct the geometry in the workspace.

  g = decsg(gd,sf,ns);

  %% Create E Model
  emagmodel = createpde("electromagnetic", "conduction");

  geometryFromEdges(emagmodel, g)
  %% Rest

  pdegplot(emagmodel,"EdgeLabels","on", "FaceLabels", "on")

  emagmodel.VacuumPermittivity = 8.8541878128E-12;
  electromagneticProperties(emagmodel, ...
      "RelativePermittivity",1.00059, ...
      "Conductivity", 1);
  %%
  electromagneticBC(emagmodel,"Voltage",1.1,"Edge",[3 4 10 11]);
  electromagneticBC(emagmodel,"Voltage",0.5,"Edge",[5 6 9 12]);
  electromagneticBC(emagmodel,"Voltage",0,"Edge",[1 2 7 8]);


  generateMesh(emagmodel);

  R = solve(emagmodel);

  pdeplot(emagmodel,"XYData",R.ElectricPotential, ...
      "Contour","on", ...
      "FlowData",[R.CurrentDensity.Jx,R.CurrentDensity.Jy])

  sample_poly = polyshape(sample_x, sample_y);
  in_poly = polyshape(in_electrode_x, in_electrode_y);

  hold on;
  plot(sample_poly);
  plot(in_poly);
  axis equal
  hold off;

\end{lstlisting}
